\documentclass[a4paper,12pt]{article}
%\usepackage[backend=biber]{biblatex}
\usepackage{cv_ozi}

\usepackage[utf8]{inputenc}
\usepackage{libertine}
%\usepackage{librebaskerville}

%\usepackage[sc]{mathpazo}
%\linespread{1.05}  

\usepackage[T1]{fontenc}

\usepackage{longtable}
%\geometry{top=18mm, inner=20mm, outer=20mm, bottom=00mm}
%\geometry{top=18mm, left=12mm, right=15mm, bottom=00mm}
%\geometry{text={17.6cm,26.3cm},centering}
%\titlespacing{\section}{0pt}{1.20ex}{0.5ex}


\name{Ozan Keysan}

\info{Address: & Institute for Energy Systems\\
	& The University of Edinburgh, EH9 3JL\\
      Phone: & +44 (0)131 6505629\\
      Email: & o.keysan@ed.ac.uk\\
      Web: & ozan.keysan.me} %\\
%      Marital status: & single}
%      WWW: & ToBeIncluded.com} 
 
\clearpage

\bibliography{ozan_publications}
\makebibcategory{national}{National Papers}
\makebibcategory{patent}{Patent}
\makebibcategory{thesis}{Thesis}
\makebibcategory{reports}{Technical Reports}
 
%\addtocategory{books}{MWH3,ITSM91,ITSM94,expsmooth08}
\addtocategory{papers}{Keysan2009,Hodgins2011a,Keysan2011b,Keysan2012a,Keysan2012e,Keysan2012d}
%\addtocategory{papers}{HDR96,HBG96,HF96,GHH97,HW97,LFSH97,GH98}
\addtocategory{conferences}{Hodgins2009,Hodgins2010,Keysan2010c,Keysan2010b,Keysan2010a,Keysan2010,Keysan2011e,Keysan2011c,Keysan2011d,Keysan2012,Echenique2013, Keysan2013a, Keysan2013b}
\addtocategory{chapters}{Keysan2013}
\addtocategory{patent}{Keysan2011g}
\addtocategory{national}{Ertan2009a,Keysan2012c,Ertan2012}
\addtocategory{thesis}{Keysan2008a}
\addtocategory{reports}{Keysan2012g,Keysan2012f,Keysan2010e,Keysan2009h,Keysan2009g,Keysan2009f,Keysan2009e}
\addtocategory{unpublished}{Mueller2013,Lara2013,Keysan}

%books 	Books
%papers 	Ref­er­eed research papers
%chap­ters 	Book chap­ters
%con­fer­ences 	Papers in con­fer­ence proceedings
%techre­ports 	Unpub­lished work­ing papers
%bookre­views 	Book reviews
%edi­to­ri­als 	Edi­to­ri­als
%phd 	PhD the­sis
%sub­pa­pers 	Sub­mit­ted papers
%cur­pa­pers 	Cur­rent projects

%It is easy to add your own categories and titles if these are not suitable. 
%For example, if you want to include posters in your CV, put the following in the preamble:

%\makebibcategory{posters}{Conference posters}
%\makebibcategory{posters}{Publications}

%\addtocategory{posters}{Portugal2012,McDonald2008f}

%After the preamble, my CV looks like this: 
\begin{document}
\maketitle
 
\section{Education \& Qualifications}
\begin{tabular}{llp{12cm}}
 2013 & PhD & Electrical Engineering, University of Edinburgh, UK\\  
& & \textit{Thesis Title: A Superconducting Generator Design for Offshore Wind Turbines} \\ 
 2008 & MSc & Electrical-Electronics Engineering, Middle East Technical University \\
& & \textit{Thesis Title: A Non-Invasive Speed and Position Sensor for Induction Machines Using External Search Coils}\\
 2005 & BSc & Electrical-Electronics Engineering, Middle East Technical University \\

\end{tabular}

\section{Experience}

\begin{tabular}{lp{3.6cm}l}
2011-- & Research Associate & Institute for Energy Systems, University of Edinburgh \\ 
2010--2012 & Electrical Machine Design Consultant & NGenTec, Edinburgh\\
2009--2010 & Research Associate & Institute for Energy Systems, University of Edinburgh\\
2005-2009 & Research Assistant & Middle East Technical University\\ 
\end{tabular}


\section{Research Interests}

My main area of interest is in the design and development of electrical machines. In particular, I am working on novel machine topologies such as superconducting machines and permanent-magnet machines. I also worked on the structural and thermal analysis of electrical machines and implementation of linear machines to wave energy converters.

\section{Research Experience \& Projects}
\begin{longtable}{lp{3cm}p{12cm}}
2011--Present & EU FP7 Project & Marina Platform Project is an EU FP7 project with 17 industrial and academic partners. I am working as a full-time researcher in the project, which aims to design combined wave and wind energy platforms. My duty is to compare different generator topologies and power take-off systems in terms of efficiency and reliability.\\
2013 & United Arab Emirates Uni. & Design of a 5 kW permanent-magnet linear generator test rig.\\
2013 & General Electric & A project to validate the superconducting generator topology that I developed during my PhD. General Electric supplied the superconducting coil and test facilities for the cryogenic tests.\\
2011 & NGenTec & Design consultant for a medium-speed generator design. Electromagnetic optimization a 5 MW, 300 rpm permanent-magnet generator has been completed.\\
2010 & NGenTec & Design and optimization of a 1 MW, 12 rpm direct-drive generator for a wind turbine. The machine has been manufactured and successfully tested.\\
2010 & SMART R\&D Grant & Design and testing of a 25 kW axial flux permanent-magnet generator. The experimental results have been used to evaluate the thermal performance of the generator and the cooling system.\\
2010 & Hayward Tyler & Feasibility analysis of a
submerged and flooded permanent-magnet generator. Thermal performance of the flooded generator has been
investigated as well as the corrosion mechanisms and fluid drag losses.\\
2009--2010 & NPower Project & The feasibility analysis of direct-drive PM generators for two wave energy converters (Aquamarine,AWS Ocean Power) and two tidal energy converters (Marine Current Turbines, Scotrenewables) have been investigated. An analytical and optimization tool is developed, and licensed by University of Edinburgh for further use.\\
2005--2008 & METU & A novel method to estimate the rotor speed and position of an induction motor using the fringing flux out
of the rotor cage is developed. An international patent has been awarded to this work.\\
\end{longtable}

\section{Teaching Experience}

\subsection{School of Engineering, University of Edinburgh}

Laboratory Supervision:
\begin{itemize}
\item Power Engineering Lab (2nd year): This course introduces students to the techniques and equipment used in the generation, transmission, distribution and utilisation of electrical power.
\item Power Generation Lab (3rd year): The lab aims to give the students experience in working with rotating machines and power electronic equipment and synchronization to the grid. 

\end{itemize}

Courses:
\begin{itemize}
\item The Industrial Doctoral Centre for Offshore Renewable Energy (four-year EngD programme), "Introduction to Superconductivity and Superconducting Generators".
\end{itemize}


Supervision:
\begin{itemize}
\item Mario Recio Lara, "Development of a mobile phone application to detect speed and faults of electrical machines", MSc Thesis, 2013.
\item Marzia Akbari, "Comparison and control of power take-off systems for combined wind/wave energy platforms", MSc Thesis, 2013.
\end{itemize}



\subsection{Middle East Technical University}
Laboratory Supervision
\begin{itemize}
\item Electromechanical Energy Conversion I-II (EE361, EE362).
\item Static Power Conversion (EE463).
\item Laboratory coordinator (4 semesters). 
\end{itemize}


\section{Honours \& Awards}
\begin{tabular}{lp{16cm}}
2013 & Staff Scholarship, University of Edinburgh. \\
2012 & Young Researcher Support, International Conference on Superconductivity and Magnetism, ICSM.\\ 
2012-2013 & PhD Overseas Fee Waiver and Stipend, University of Edinburgh. \\
2011 & Young Researcher Award and Travel Grant, European Conference on Applied Superconductivity, EUCAS. \\ 
2011 & IEEE Membership and Travel Grant, IEEE Power Electronics Society. \\ 
2011 & Best Poster Award, IEEE International Electric Machines and Drives Conference, IEMDC. \\
2010 & Best Paper Award, IEEE International Conference on Electrical Machines, ICEM. \\
2010-2012 & PhD Scholarship on Renewable Energy, Hopewell Holdings, Hong Kong. \\
2005--2007 & TUBITAK Graduate Fellowship. \\
2005 & Ranked 2nd in the Academic Personnel and Graduate Education Exam (ALES).\\
2005 & Finalist in the METU Entrepreneurship Competition (Yeni İşler Yeni Fikirler).\\
2001--2005 & Dean's high honour list (3 times), Dean's honour list (3 times). \\
\end{tabular}


\section{Courses \& Seminars}
\begin{itemize}
\item Future Reliable Renewable Energy Conversion Systems, 4th Flagship Seminar, Chongqing, China, 2012.
\item Supervising Postgraduate Research, Iain Davidson, 2013. 
\item Large-Scale Parallel Computing, University of Edinburgh, 2013.
\item Superconducting Machines, UK Magnetics Society, University of Oxford, 2012.
\item Thermal and Mechanical Aspects of High Performance Electrical Machines, UK Magnetics Society, 2011.
\item LaTeX for scientific publications, Skills Development Edinburgh, 2010.
\item Opera 2D/3D FEA Analysis of Electrical Machines, Cobham Ltd. Vector Fields, 2009.
\item Marine and Tidal Energy Workshop, Industrial Problems in Marine Energy Workshop, 2009.
\item Project Management for Researchers, Fistral Training \& Consultancy Ltd., 2009.
\item Project Presentation on International Venture Capital Forum, Athens, 2006.
\item Project Management and Entrepreneurship Course, 30 days, METU Technopolis, 2005.
\end{itemize}


\section{Memberships}
\begin{tabular}{ll}
2011--Present & IEEE Member \\
2011--Present & IEEE Power and Energy Society Member \\
2010--Present & UK Magnetics Society Member\\
2005--Present & TMMOB Electrical Engineers Society Member\\
\end{tabular}

\section{Academic Work}
\begin{tabular}{ll}
Reviews for & IEEE Transactions of Industrial Electronics \\
& IEEE Transactions of Industrial Informatics \\
& IET Renewable Power Generation Journal \\
& IEEE International Conference on Electrical Machines and Drives \\
& IET Power Electronics, Machines and Drives Conference \\
& IEEE International Conference on Electrical Machines\\
\end{tabular}


\section{Software}
\begin{itemize}
\item Programming: Python, R, Matlab, C++.
\item Finite Element Analysis: Opera, FEMM, ANSYS, Gmsh, GetDP.
\item Computer Aided Design: SolidWorks, AutoCad.
\item Computational Fluid Dynamics: OpenFOAM.
\end{itemize}

\section{Activities \& Hobbies}
\begin{tabular}{ll}
Mountaineering & Active Member of METU Mountaineering Club\\
Sailing & Dinghy and Yacht Sailing, Certificate of Competence For Operators of Pleasure Craft. \\
Blogging & asuyatuyolar.org\\
\end{tabular}

\section{Languages}
\begin{tabular}{lll} 
English (Fluent), & Spanish (Intermediate), & Chinese (Beginner)
\end{tabular}


\begin{publications}
%\printbib{posters}
%\printbib{books}
\printbib{papers}
\printbib{chapters}
\printbib{patent}
\printbib{conferences}
\printbib{national}
\printbib{reports}
\printbib{thesis}
\printbib{unpublished}
%\printbib{techreports}
%\printbib{bookreviews}
%\printbib{editorials}
\end{publications}

%Each \printbib command will add a section with all the publications in that cateegory,
%listed in chronological order and sorted by name within each year.
%The total number of publications listed inside the publications environment is 
%calculated and the page numbers for the publications sections are stored. 
%So I have the following line in the Research section of my CV:

%I have authored \ref{sumpapers} papers, chapters or books on statistical topics. A list of these appears on pages \pageref{papersstart}--\pageref{papersend}.

\end{document}
