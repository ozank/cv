\documentclass[a4paper,12pt]{article}
%\usepackage[backend=biber]{biblatex}
\usepackage{cv_ozi}

\usepackage[utf8]{inputenc}
\usepackage{libertine}
%\usepackage{librebaskerville}

%\usepackage[sc]{mathpazo}
%\linespread{1.05}  

\usepackage[T1]{fontenc}

\usepackage{longtable}
%\geometry{top=18mm, inner=20mm, outer=20mm, bottom=00mm}
%\geometry{top=18mm, left=12mm, right=15mm, bottom=00mm}
%\geometry{text={17.6cm,26.3cm},centering}
%\titlespacing{\section}{0pt}{1.20ex}{0.5ex}


\name{Ozan Keysan}

\info{Address: & Orta Doğu Teknik Üniversitesi\\
	& Elektrik-Elektronik Müh., 06800, Ankara\\
      Phone: & +90 312 210 7586\\
      Email: & keysan@metu.edu.tr\\
      Web: & http://keysan.me} %\\
 
\clearpage

\bibliography{ozan_publications}
\makebibcategory{national}{National Papers}
\makebibcategory{patent}{Patent}
\makebibcategory{theses}{Theses}
\makebibcategory{reports}{Technical Reports}
 
%\addtocategory{books}{MWH3,ITSM91,ITSM94,expsmooth08}
\addtocategory{papers}{Keysan2009,Hodgins2011a,Keysan2011b,Keysan2012a,Keysan2012e,Keysan2012d,Keysan2014e,Keysan2014f,Temiz2019,TARVIRDILUASL2019,Ceylan2019,
Zeinali2019,Ugur2019,Mueller2019,Ugur2018a,Ceylan2018,Radyjowski2016a}
%\addtocategory{papers}{HDR96,HBG96,HF96,GHH97,HW97,LFSH97,GH98}

\addtocategory{conferences}{Hodgins2009,Hodgins2010,Keysan2010c,Keysan2010b,Keysan2010a,Keysan2010,Keysan2011e,Keysan2011c,Keysan2011d,Keysan2012,Echenique2013, Keysan2013a,Keysan2013b,Keysan2014,Keysan2014a,Burchell2014,Macadre2015,Ertekin2019,Cakal2019,Sahin2018,Ugur2018c,Cakal2018a,Ugur2018,Mutlu2018,
Duymaz2018,Ugur2018b,Akgemci,Karakaya2018b,Ceylan2017i,Zeinali2017,Ceylan2017b,Ugur2017,Bernholz2016,Alemdar2016,Inanr2015,Mcdonald2015}
\addtocategory{chapters}{Keysan2013}
\addtocategory{patent}{Keysan2011g}
\addtocategory{national}{Ertan2009a,Keysan2012c,Ertan2012}
\addtocategory{theses}{Keysan2008a,Keysan2014c}
\addtocategory{reports}{Keysan2012g,Keysan2012f,Keysan2010e,Keysan2009h,Keysan2009g,Keysan2009f,Keysan2009e}
	

%books 	Books
%papers 	Ref­er­eed research papers
%chap­ters 	Book chap­ters
%con­fer­ences 	Papers in con­fer­ence proceedings
%techre­ports 	Unpub­lished work­ing papers
%bookre­views 	Book reviews
%edi­to­ri­als 	Edi­to­ri­als
%phd 	PhD the­sis
%sub­pa­pers 	Sub­mit­ted papers
%cur­pa­pers 	Cur­rent projects

%It is easy to add your own categories and titles if these are not suitable. 
%For example, if you want to include posters in your CV, put the following in the preamble:

%\makebibcategory{posters}{Conference posters}
%\makebibcategory{posters}{Publications}

%\addtocategory{posters}{Portugal2012,McDonald2008f}

%After the preamble, my CV looks like this: 
\begin{document}
\maketitle

\section{Experience}

\begin{tabular}{lp{3.6cm}l}
2014--Present & Assistant Professor & \href{http://eee.metu.edu.tr}{Department of Electrical and Electronics Engineering} \\ 
 &  & \href{http://metu.edu.tr}{Middle East Technical University}, Ankara, Turkey  \\ 
2011--2014 & Research Associate & \href{https://www.eng.ed.ac.uk/research/institutes/ies}{Institute for Energy Systems}, \href{https://www.ed.ac.uk}{University of Edinburgh}  \\ 
2010--2012 & Electrical Machine Design Consultant & NGenTec, Edinburgh\\
2009--2010 & Research Associate & Institute for Energy Systems, University of Edinburgh\\
2005-2009 & Research Assistant & Middle East Technical University\\ 
\end{tabular}
 
\section{Education \& Qualifications}
\begin{tabular}{llp{12cm}}
 2013 & PhD & Electrical Engineering, University of Edinburgh, UK\\  
& & \textit{Thesis Title: Superconducting Generators for Large Offshore Wind Turbines} \\ 
 2008 & MSc & Electrical-Electronics Engineering, Middle East Technical University \\
& & \textit{Thesis Title: A Non-Invasive Speed and Position Sensor for Induction Machines Using External Search Coils}\\
 2005 & BSc & Electrical-Electronics Engineering, Middle East Technical University \\

\end{tabular}

\section{Academic \& Administrative Work}

\begin{itemize}

\item Vice Chair, Dept. of Electric-Electronics Engineering, 2018-2019.
\item Advisor to Dean, 2016-2018.
\item Member of the Executive Board, \href{http://ruzgem.metu.edu.tr/}{METU-Wind Center}, 2016-Present
\item Member of the Executive Board, \href{http://tdi.metu.edu.tr}{Design Factory}, 2016-2017.
\item Associate Editor, \href{http://digital-library.theiet.org/content/journals/iet-rpg}{IET Renewable Power Generation}, 2016-Present.
\item EU \href{http://www.cost.eu/}{Cost Action} Rapporteur, 2017

\end{itemize}

\section{Research Interests}

My main area of interests are renewable energy generation, smart-grids and design of electrical machines. In particular, I am working on novel electrical machine topologies such as integrated modular motor drives, superconducting machines and permanent-magnet machines. I also worked on the implementation of direct-drive systems to renewable energy systems such as wind, wave and tidal energy converters.

\clearpage

\section{Research Experience \& Projects}
\begin{longtable}{lp{3cm}p{12cm}}
2018--2019 & USTDA & USA, Trade and Development Agency, \textbf{METU Smart Campus Project}: This project is the only funded project from the \href{https://docs.wixstatic.com/ugd/96b21c_3c398c840786434e9a3362c733396559.pdf}{Smart Cities call} of 2016. In the project, feasibility analysis of smart city concepts for METU will be investigated. These include roof-top PVs, electric shuttle bus, electric bike network, smart LED lighting, combined heat and power generation facilities. The project aims to make METU campus a sustainable, self-sufficient green campus. Project has been \href{https://www.ustda.gov/news/press-releases/2017/ustda-supports-metu-smart-campus-technical-assistance-grant}{signed} in September 2017, and it is expected to be completed in April 2019. Project budget: \$830.000. \\

2017--2018 & TUBITAK 3501 & \textbf{Development of a 7,5 kW Permanent Magnet Integrated Modular Motor Drive System}: The aim of this project is to develop an IMMD system where the electric motor and its drive are integrated into a single package. The motor and drive will consist of several modules to increase fault tolerance. Gallium Nitride (GaN) power semiconductor devices will be used to reduce the size and increase the efficiency. Project budget: 177.000 TL.\\

2017--2019 & ASELSAN SST & \textbf{Electromagnetic Design and Optimization of a 10 MJ Electromagnetic Launcher - Phase II}: Second phase of the \href{http://www.millisavunma.com/aselsan-tufan-elektromanyetik-top-sistemi/}{TUFAN} railgun project in which a 10 MJ prototype will be tested. Detailed 3D FEA models are developed to obtain electrical, mechanical force, and thermal characteristics of the railgun. Project budget: 392.000 TL.\\

2017--2018 & ASELSAN MGEO & \textbf{Design of a high performance linear servo actuator and PCB Motor}: Two high performance actuators (linear and axial flux permanent magnet) will be designed and manufactured in this project. The aim is to manufacture these actuators in Turkey instead of exporting. Project budget: 198.000 TL.\\

2017--2018 & EnerjiSa-H2020 Project & \textbf{Pattern Recognition in Power Systems}: The aim of \href{http://europeanpatternrecognition.eu/}{EPR Project} is to implement pattern recognition techniques for power systems to detect faults and plan predictive maintenance on system components. METU’s work packages consist of \href{https://www.youtube.com/watch?v=_52WUWsKuXA&feature=youtu.be}{synthetic inertia implementation} in wind turbines to improve power system stability and analyze distribution level power quality to detect anomalies. Total Project budget: € 2.400.000.\\

2017 & ASELSAN UGES & \textbf{Feasibility analysis and design of a 2.5MW direct-drive wind turbine generator}: In this project techno-economic analysis of different generator technologies are performed to find the most suitable generator technology for ASELSAN. A detailed electromagnetic, structural and thermal optimization of a direct-drive generator using FEA tools is presented. Project budget: 87.000TL.\\

2016--2017 & ASELSAN SST & \textbf{Electromagnetic Design and Optimization of a 200kJ-2MJ Electromagnetic Launcher}: In this project electromagnetic and thermal analyses are performed on the \href{http://www.millisavunma.com/aselsan-tufan-elektromanyetik-top-sistemi/}{TUFAN} railgun project. Detailed 3D FEA models are developed to obtain electrical, mechanical force, and thermal characteristics of the railgun. Project budget: 397.000 TL.\\

2016 & METU BAP & \textbf{Design of a Modular Data Acquisition Board}.\\

2016 & ANDAR Servo & \textbf{Design and Manufacturing of a Eddy Current Damper for Aerospace Application}.\\

2016 & ANDAR Servo & \textbf{Design and Manufacturing of a Brushless PM Servo Motor}.\\

2016 & Newton Fund & \textbf{Design of a superconducting wind turbine prototype}.\\

2016 & TUBITAK-2232 & \textbf{Design of modular and fault-tolerant generators for direct-drive wind turbines}.\\

2014 & NAREC & \textbf{Design of next generation HVDC network for the offshore renewable energy industry}. My role in the project is to design a high-frequency high-power transformer that can be coupled to the HVDC transmission line.\\

2011--2014 & EU FP7 Project & \textbf{Marina Platform Project} is an EU FP7 project with 17 industrial and academic partners. I am working as a full-time researcher in the project, which aims to design combined wave and wind energy platforms. I've led the work package in the University of Edinburh, the aim of which is to compare different generator topologies and power take-off systems in terms of efficiency and reliability.\\
2013 & United Arab Emirates Uni. & \textbf{Design of a 5 kW permanent-magnet linear generator test rig}.\\
2013 & General Electric & A project to validate the \textbf{superconducting generator} topology that I developed during my PhD. General Electric agreed to loan the superconducting coil, vacuum chamber and other test equipments for the cryogenic experiments.\\
2011 & NGenTec & Design consultant for a \textbf{medium-speed PM generator design}. Electromagnetic optimization a 5 MW, 300 rpm permanent-magnet generator has been completed.\\
2010 & NGenTec & \textbf{Design and optimization of a 1 MW, 12 rpm direct-drive generator for a wind turbine}. The machine has been manufactured and successfully tested.\\
2010 & SMART R\&D Grant & \textbf{Design and testing of a 25 kW axial flux permanent-magnet generator}. The experimental results have been used to evaluate the thermal performance of the generator and the cooling system.\\
2010 & Hayward Tyler & \textbf{Feasibility analysis of a
submerged and flooded permanent-magnet generator}. Thermal performance of the flooded generator has been
investigated as well as the corrosion mechanisms and fluid drag losses.\\
2009--2010 & NPower Project & \textbf{Feasibility analysis of direct-drive PM generators for two wave energy converters} (Aquamarine,AWS Ocean Power) and two tidal energy converters (Marine Current Turbines, Scotrenewables) have been investigated. An analytical and optimization tool is developed, and licensed by University of Edinburgh for further use.\\

\end{longtable}

\section{Teaching Experience}

\subsection{Middle East Technical University}

For a full list of courses that I taught please visit my \href{http://keysan.me/courses}{courses} webpage. 

\begin{itemize}
\item Electromechanical Energy Conversion I--II (EE361, EE362).
\item Static Power Conversion I--II (EE463, EE464).
\item Utilization of Electral Energy (EE462).
\item Design of Electrical Machines (EE564).
\item Collaborative Design Studio (ID403)
\end{itemize}


\subsection{School of Engineering, University of Edinburgh}

Laboratory Supervision:
\begin{itemize}
\item Power Engineering Lab (2nd year): This course introduces students to the techniques and equipment used in the generation, transmission, distribution and utilisation of electrical power.
\item Power Generation Lab (3rd year): The lab aims to give the students experience in working with rotating machines and power electronic equipment and synchronization to the grid. 

\end{itemize}

Courses:
\begin{itemize}
\item The Industrial Doctoral Centre for Offshore Renewable Energy (four-year EngD programme), "Introduction to Superconductivity and Superconducting Generators".
\end{itemize}




\section{Honours \& Awards}
\begin{tabular}{lp{16cm}}
2018 & Educator of the Year Award, Parlar Foundation \\
2018 & Best Lecturer, 2017-2018 Spring semester (based on student evaluations), METU. \\
2017 & Best Lecturer, 2017-2018 Fall semester (based on student evaluations), METU. \\
2017 & Best Poster Award, Renewable Energy and Energy Efficiency Conference, DAAD. \\
2013 & Staff Scholarship, University of Edinburgh. \\
2012 & Young Researcher Support, International Conference on Superconductivity and Magnetism, ICSM.\\ 
2012-2013 & PhD Overseas Fee Waiver and Stipend, University of Edinburgh. \\
2011 & Young Researcher Award and Travel Grant, European Conference on Applied Superconductivity, EUCAS.\\ 
2011 & IEEE Membership and Travel Grant, IEEE Power Electronics Society. \\ 
2011 & Best Poster Award, IEEE International Electric Machines and Drives Conference, IEMDC. \\
2010 & Best Paper Award, IEEE International Conference on Electrical Machines, ICEM. \\
2010-2012 & PhD Scholarship on Renewable Energy, Hopewell Holdings, Hong Kong. \\
2005--2007 & TUBITAK Graduate Fellowship. \\
2005 & Ranked 2nd in the Academic Personnel and Graduate Education Exam (ALES).\\
2005 & Finalist in the METU Entrepreneurship Competition (Yeni İşler Yeni Fikirler).\\
2001--2005 & Dean's high honour list (3 times), Dean's honour list (3 times). \\
\end{tabular}


\section{Courses \& Seminars}
\begin{itemize}

\item \href{https://hpi.de/school-of-design-thinking.html}{HPI}, Design Thinking Workshop, 3 days, 2016.
\item \href{https://designfactory.aalto.fi/}{Aalto Design Factory}, Design factory crash course, 2 days, 2015.
\item Future Reliable Renewable Energy Conversion Systems, 4th Flagship Seminar, Chongqing, China, 2012.
\item Supervising Postgraduate Research, Iain Davidson, 2013. 
\item Large-Scale Parallel Computing, University of Edinburgh, 2013.
\item Superconducting Machines, UK Magnetics Society, University of Oxford, 2012.
\item Thermal and Mechanical Aspects of High Performance Electrical Machines, UK Magnetics Society, 2011.
\item LaTeX for scientific publications, Skills Development Edinburgh, 2010.
\item Opera 2D/3D FEA Analysis of Electrical Machines, Cobham Ltd. Vector Fields, 2009.
\item Marine and Tidal Energy Workshop, Industrial Problems in Marine Energy Workshop, 2009.
\item Project Management for Researchers, Fistral Training \& Consultancy Ltd., 2009.
\item Project Presentation on International Venture Capital Forum, Athens, 2006.
\item Project Management and Entrepreneurship Course, 30 days, METU Technopolis, 2005.
\end{itemize}


\section{Memberships}
\begin{tabular}{ll}
2011--Present & IEEE Member \\
2011--Present & IEEE Power and Energy Society Member \\
2010--Present & UK Magnetics Society Member\\
2005--Present & TMMOB Electrical Engineers Society Member\\
\end{tabular}


\section{Software}
\begin{itemize}
\item Programming: Python, R, Matlab, C++.
\item Finite Element Analysis: Opera, FEMM, ANSYS, Gmsh, GetDP.
\item Computer Aided Design: SolidWorks, AutoCad.
\item Computational Fluid Dynamics: OpenFOAM.
\end{itemize}

\section{Activities \& Hobbies}
\begin{tabular}{ll}
Mountaineering & Active Member of METU Mountaineering Club\\
Sailing & Dinghy and Yacht Sailing, Certificate of Competence For Operators of Pleasure Craft. \\
Blogging & asuyatuyolar.org\\
\end{tabular}

\section{Languages}
\begin{tabular}{lll} 
English (Fluent), & Spanish (Intermediate), & Chinese (Beginner)
\end{tabular}



\begin{publications}
%\printbib{posters}
%\printbib{books}
\printbib{papers}
\printbib{chapters}
\printbib{patent}
\printbib{conferences}
\printbib{national}
\printbib{reports}
\printbib{theses}

\section{Theses Supervised}


\begin{enumerate}

\item Lütfi Boyacı, Proton Irradiation and Gamma Ray Irradiation Testing Studies on the Commercial Grade GaNFETs to Investigate their Characteristics under the Space Radiation Environment, MSc 2019
\item Aykut Demirel, Condition Monitoring and Fault Diagnosis of Electrical Motor Drive Systems,  MSc 2019
\item Mehmet Kaan Mutlu, Limited-jerk Sinusoidal Trajectory Design for Field Oriented Control of Permanent Magnet Synchronous Motor With H-infinity Optimal Controller, MSc 2019
\item Hakan Temiz, Grid Impedance Estimation Based Adaptive Controller Design for Back-to-Back Wind Turbine Power Converters for Stable Operation in Distorted and Weak Grid, MSc 2019
\item Erencan Duymaz, Investigation of Inertial Support Limits in Wind Turbines and the Effects in the Power System Stability, MSc 2019
\item Aysel Akgemci, Hybrid Excited Synchrnous Generator Design and Comparison of Direct Drive Wind Turbines, MSc 2019
\item Doğa Ceylan, Electromagnetic Simulation and Optimization of an Electromagnetic Launcher, MSc 2018.
\item Mert Bildirici, Reducing the Cost of Electric Transmission and Distribution Systems with Wind Generation by Means of Energy Storage and Demand Side Management, MSc 2018.
\item Aydın Başkaya, Design of a Modular, Axial-flux Direct Drive Permanent
Magnet Generator For Wind Turbines, MSc 2018.
\item Melih Var, Improvement of the Electric Field Distribution in the Medium Voltage Gas Insulated Inductive Voltage Transformer, MSc 2017.
\item Öztürk Şahin Alemdar, Design and Implementation of an Unregulated DC/DC Transformer Module Using LLC Resonant Converter, MSc 2016. 
\item Mario Recio Lara, Development of a mobile phone application to detect speed and faults of electrical machines*, MSc 2013.
\item Marzia Akbari, Comparison and control of power take-off systems for combined wind/wave energy platforms, MSc 2013.

\end{enumerate}

%\printbib{unpublished}
%\printbib{techreports}
%\printbib{bookreviews}
%\printbib{editorials}
\end{publications}

%Each \printbib command will add a section with all the publications in that cateegory,
%listed in chronological order and sorted by name within each year.
%The total number of publications listed inside the publications environment is 
%calculated and the page numbers for the publications sections are stored. 
%So I have the following line in the Research section of my CV:

%I have authored \ref{sumpapers} papers, chapters or books on statistical topics. A list of these appears on pages \pageref{papersstart}--\pageref{papersend}.

\end{document}
