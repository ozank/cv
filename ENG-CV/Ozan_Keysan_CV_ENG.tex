\documentclass[a4paper,12pt]{article}
%\usepackage[backend=biber]{biblatex}
\usepackage{cv_ozi}

\usepackage[utf8]{inputenc}
\usepackage{libertine}
%\usepackage{librebaskerville}

%\usepackage[sc]{mathpazo}
\linespread{1.1}  

\usepackage[T1]{fontenc}

\usepackage{longtable}
%\geometry{top=18mm, inner=20mm, outer=20mm, bottom=00mm}
%\geometry{top=18mm, left=12mm, right=15mm, bottom=00mm}
%\geometry{text={17.6cm,26.3cm},centering}
%\titlespacing{\section}{0pt}{1.20ex}{0.5ex}

\usepackage{hyperref}
\urlstyle{rm}

\name{Ozan Keysan}

\info{Address: & Institute for Energy Systems\\
	& The University of Edinburgh, EH9 3JL\\
      Phone: & +90 534 743 9275\\
      		 & +44 (0)131 6505629\\
      Email: & o.keysan@ed.ac.uk\\
      Web: & \url{ozan.keysan.me}} %\\
 
\clearpage

\bibliography{ozan_publications}
\makebibcategory{national}{National Papers (in Turkish)}
\makebibcategory{patent}{Patent}
\makebibcategory{thesis}{Thesis}
\makebibcategory{reports}{Technical Reports}
\makebibcategory{conference-abstract}{Conference Papers: Abstract Referreed}
 
%\addtocategory{books}{MWH3,ITSM91,ITSM94,expsmooth08}
\addtocategory{papers}{Keysan2009,Hodgins2011a,Keysan2011b,Keysan2012a,Keysan2012d,Keysan2012e}
%\addtocategory{papers}{HDR96,HBG96,HF96,GHH97,HW97,LFSH97,GH98}
\addtocategory{conferences}{Hodgins2010,Keysan2010,Keysan2010a,Keysan2010b,Keysan2011c,Keysan2011d,Echenique2013,Ertan2013, Keysan2013b,Keysan2012,Keysan2011e}
\addtocategory{conference-abstract}{Hodgins2009,Keysan2013a}
\addtocategory{chapters}{Keysan2013,Keysan2014b}
\addtocategory{patent}{Keysan2011g}
\addtocategory{national}{Keysan2012c}
\addtocategory{thesis}{Keysan2008a,Keysan2014c}
%\addtocategory{reports}{Keysan2012g,Keysan2012f,Keysan2010e,Keysan2009h,Keysan2009g,Keysan2009f,Keysan2009e}
%\addtocategory{unpublished}{Mueller2013,Lara2013}

%books 	Books
%papers 	Ref­er­eed research papers
%chap­ters 	Book chap­ters
%con­fer­ences 	Papers in con­fer­ence proceedings
%techre­ports 	Unpub­lished work­ing papers
%bookre­views 	Book reviews
%edi­to­ri­als 	Edi­to­ri­als
%phd 	PhD the­sis
%sub­pa­pers 	Sub­mit­ted papers
%cur­pa­pers 	Cur­rent projects

%It is easy to add your own categories and titles if these are not suitable. 
%For example, if you want to include posters in your CV, put the following in the preamble:

%\makebibcategory{posters}{Conference posters}
%\makebibcategory{posters}{Publications}

%\addtocategory{posters}{Portugal2012,McDonald2008f}

%After the preamble, my CV looks like this: 
\begin{document}
\maketitle
 
\section{Degrees Awarded}
\begin{tabular}{llp{14cm}}
 2013 & PhD & Electrical Engineering, University of Edinburgh, UK\\  
& & Dissertation Title: \textit{Superconducting Generators for Large Offshore Wind Turbines} \\ 
 2008 & MSc & Electrical-Electronics Engineering, Middle East Technical University, Turkey \\
& & Thesis Title: \textit{A Non-Invasive Speed and Position Sensor for Induction Machines Using External Search Coils}\\
 2005 & BSc & Electrical-Electronics Engineering, Middle East Technical University (METU), Turkey \\

\end{tabular}

\section{Career Since Graduation}

\begin{tabular}{lp{3.6cm}l}
2011--2014 & Research Associate & Institute for Energy Systems, University of Edinburgh \\ 
2010--2012 & Design Consultant & NGenTec, Edinburgh\\
2009--2010 & Research Associate & Institute for Energy Systems, University of Edinburgh\\
2005-2009 & Research Assistant & Middle East Technical University, Turkey\\ 
\end{tabular}


\section{Major Research Interests}

\subsection{Research Impact \& Achievements}
My research output has resulted in high research impact in terms of consultancies and knowledge exchange activities. Within last five years, I published seven journal papers (one included in RAE), two invited book contributions, 11 refereed conference papers (two won the best paper/poster award) and filed one patent. 

The superconducting machine design I developed in my PhD project resulted in three invited talks from Institute of Physics, UK Magnetics Society and Superconducting for Energy. Also I got the Young Researcher Award for my work in EUCAS and ICSM conferences. Furthermore, the prototype I developed improved the links between University of Edinburgh (UoE) and General Electric Energy and Power Conversion and led to transfer of an extensive superconducting lab to UoE. 

I worked with different renewable companies across UK, which helped to develop network between industry and university. I worked as the main researcher in an EU FP7 project work package, which brought £320k to UoE. 
%I represented UoE in general project assemblies and work package meetings.

%I designed several machines and power take-off systems varying from 5 kW to 5 MW, all in forms of consultancies and research partnerships, bringing research grants to IES. The details of these works are provided in the next section.

\clearpage
\subsection{Research Interests}

My main area of interest is in the design and development of electrical machines and power electronics. In particular, I am working on novel machine topologies such as superconducting machines and direct-drive permanent-magnet machines, including linear generators. I also worked in reliability assessment and control systems of renewable energy devices.

\begin{itemize}
\item \textbf{Superconducting Machines}: I designed a novel generator topology that surpasses all the existing superconducting machine designs in terms of modularity, reliability and significant cost benefits by utilising only 10\% of the SC tape compared to other designs. The topology has chance to be implemented as the world's first superconducting wind turbine. There are many research opportunities in this area that I can contribute with my research experience. SC transmission lines and fault current limiters can be excellent applications for future HVDC systems. I am also planning to work on SC magnetic gearboxes, which will be the world\u2019s first with unmatched torque density capabilities.

\item \textbf{Direct-Drive Permanent Magnet Generators}: I have been working in this area since 2009. I developed design and optimization tools for C-Gen generator, which was the main idea behind NGenTec (a spin-out company from IES). UoE has licensed my design tools  to NGenTec, and they have been used in other consultancy projects. I designed 25 kW and 1 MW generators, both of them are manufactured and tested successfully. I designed a 50 kW linear generator for the Carbon Trust Project, which led to best paper award in ICEM-2010 and 20 citations.

\item \textbf{Power Electronics and Electric Machine Drives}: I completed my Masters on control of AC motors. I developed a novel method that led to an international patent, two journal papers and many conference papers. I am currently involved in a project on HVDC subsea transmission system with NAREC. The concept reduces the number of power electronic components,and hence the overall cost. I am also involved in control of electric machines in the MARINA project, in which I compared different power aggregation techniques for combined wind and wave energy systems.

\item \textbf{Reliability and Condition Monitoring}: The reliability for renewable energy systems are critical, hence almost all the projects I have involved has reliability as the main concept. I believe there is much work to do to cope with harsh operating conditions of renewable energy devices. My PhD topic introduced a fault-tolerant, modular generator topology, which can minimise down-times in a wind turbine. I have also analysed the failure modes of combined wind and wave energy plaftorms for the MARINA Project. One of my MSc students developed an electric machine fault-detection app for smart phones, which has been downloaded 500 times less than in six months.
\end{itemize}

\clearpage

\section{Research Experience \& Projects}
\begin{longtable}{lp{3cm}p{12cm}}
2014 & NAREC & Design of next generation HVDC network for the offshore renewable energy industry. I am desigining a high-frequency high-power transformer that can be coupled to the HVDC transmission line. The proposed system has fewer power electronics and cost advantages over existing AC and DC systems. (WP Budget:£30k)\\
2011--2014 & EU FP7 Project &  MARINA Platform Project is a 3 year EU FP7 project with 17 industrial and academic partners, which aims to design combined wave and wind energy platforms. I was the main researcher in WP7, and I have led and presented UoE in work package meetings and general assemblies, which helped me to have contacts across Europe and build my presentation skills. I developed a design tool for sizing and efficiency calculation for wind and wave power take-off systems and the tools was shared with the research community as an open source tool. (WP~Budget:£300k)\\
2013 & United Arab Emirates Uni. & Design of a 5 kW permanent-magnet linear generator test rig.\\
2013 & General Electric & A project to validate the superconducting generator topology that I developed during my PhD. General Electric agreed to loan the superconducting coil, vacuum chamber and other test equipments to UoE for the cryogenic experiments.\\
2011 & NGenTec & Consultany for a medium-speed generator design. Electromagnetic optimization a 5 MW, 300 rpm permanent-magnet generator has been completed.\\
2010 & NGenTec & (Company funded £2M). Design and optimization of a 1 MW, 12 rpm direct-drive generator for a wind turbine. The machine has been manufactured and successfully tested.\\
2010 & SMART R\&D Grant & (Budget:£93k). Design and testing of a 25 kW axial flux permanent-magnet generator. The experimental results have been used to evaluate the thermal performance of the generator and the cooling system.\\
2010 & Hayward Tyler & RenewNet project. Feasibility analysis of a submerged and flooded permanent-magnet generator. Thermal performance of the flooded generator was investigated as well as the corrosion mechanisms and fluid drag losses.\\
2009--2010 & NPower Project & (Budget:£78k). The feasibility analysis of direct-drive PM generators for two wave energy converters (Aquamarine,AWS Ocean Power) and two tidal energy converters (Marine Current Turbines, Scotrenewables) were investigated. An analytical and optimization tool was developed, and licensed by University of Edinburgh for further use.\\
2005--2008 & METU & A novel method to estimate the rotor speed and position of an induction motor using the fringing flux out of the rotor cage was developed. An international patent was awarded to this work.\\
\end{longtable}

\clearpage

\section{Teaching Experience}

Since I got BSc degree in 2005, I have taught in different courses. I was a research and teaching assistant in METU for four years. I tutored in 3\textsuperscript{rd} year and 4\textsuperscript{th} year undergraduate machines and power electronics courses.

In UoE, I supervised 2\textsuperscript{nd} year Power Engineering and 3\textsuperscript{rd} year Power Generation Labs and gave lectures on IDCORE programme in the last two years. I am confident and passionate in teaching and e-learning and I can take responsibility on power electronics, power systems and machines courses as supervising. I supervised two MSc and two MEng students so far, and assisting in a few PhD projects within the group.

\subsection{School of Engineering, University of Edinburgh}

\textbf{Laboratory Supervision:}
\begin{itemize}
\item Power Engineering Lab (2nd year): This course introduces students to the techniques and equipment used in the generation, transmission, distribution and utilisation of electrical power. I have supervised in this lab since 2012.
\item Power Generation Lab (3rd year): The lab aims to give the students experience in working with rotating machines and power electronic equipment and synchronization to the grid. I have supervised in this lab since 2012.

\end{itemize}

\textbf{Lectures:}
\begin{itemize}
\item The Industrial Doctoral Centre for Offshore Renewable Energy (four-year EngD programme), "Introduction to Superconductivity and Superconducting Generators".
\end{itemize}


\textbf{Supervision:}
\begin{itemize}
\item Dauriuz Olczak, ``Structural Design of a High Temperature Superconducting Generator
'', MEng Thesis, 2012.
\item Patryk Radyjowski, ``Designing a Cooling System for a Superconductive Coil'', MEng Thesis, 2013.
\item Mario Recio Lara, ``Development of a mobile phone application to detect speed and faults of electrical machines'', MSc Thesis, 2013.
\item Marzia Akbari, ``Comparison and control of power take-off systems for combined wind/wave energy platforms'', MSc Thesis, 2013.
\end{itemize}


\subsection{Middle East Technical University}

\textbf{Laboratory Supervision:}
\begin{itemize}
\item 3\textsuperscript{rd} year Electromechanical Energy Conversion I-II (EE361, EE362).
\item 4\textsuperscript{th} year Static Power Conversion (EE463).
\item Laboratory coordinator (4 semesters). 
\end{itemize}


\section{Honours \& Awards}
\begin{tabular}{lp{16cm}}
2013 & Staff Scholarship, University of Edinburgh. \\
2012 & Young Researcher Support, International Conference on Superconductivity and Magnetism, ICSM.\\ 
2012-2013 & PhD Overseas Fee Waiver and Stipend, University of Edinburgh. \\
2011 & Young Researcher Award and Travel Grant, European Conference on Applied Superconductivity, EUCAS.\\ 
2011 & IEEE Membership and Travel Grant, IEEE Power Electronics Society. \\ 
2011 & Best Poster Award, IEEE International Electric Machines and Drives Conference, IEMDC. \\
2010 & Best Paper Award, IEEE International Conference on Electrical Machines, ICEM. \\
2010-2012 & PhD Scholarship on Renewable Energy, Hopewell Holdings, Hong Kong. \\
2005--2007 & TUBITAK Graduate Fellowship. \\
2005 & Ranked 2nd in the Academic Personnel and Graduate Education Exam (ALES).\\
2005 & Finalist in the METU Entrepreneurship Competition (Yeni İşler Yeni Fikirler).\\
2001--2005 & Dean's high honour list (3 times), Dean's honour list (3 times). \\
\end{tabular}

\section{Invited Presentations \& Lectures}
\begin{itemize}
\item Superconducting for Energy Conference, Pasteum, Italy, 2014.
\item Current Research in Large-Scale Applications of Superconductivity, Institute of Physics, London, 2013.
\item Superconducting Machines, UK Magnetics Society, University of Oxford, 2012.
\item Future Reliable Renewable Energy Conversion Systems, 4th Flagship Seminar, Chongqing, China, 2012.
\item Thermal and Mechanical Aspects of High Performance Electrical Machines, UK Magnetics Society, 2011.
\item Marine and Tidal Energy Workshop, Industrial Problems in Marine Energy Workshop, 2009.
\end{itemize}

\section{Seminars}
\begin{itemize}
\item Supervising Postgraduate Research Seminar, Iain Davidson, 2013. 
\item Large-Scale Parallel Computing, University of Edinburgh, 2013.
\item LaTeX for scientific publications, Skills Development Edinburgh, 2010.
\item Opera 2D/3D FEA Analysis of Electrical Machines, Cobham Ltd. Vector Fields, 2009.
\item Project Management for Researchers, Fistral Training \& Consultancy Ltd., 2009.
\item Project Presentation on International Venture Capital Forum, Athens, 2006.
\item Project Management and Entrepreneurship Course, 30 days, METU Technopolis, 2005.
\end{itemize}


\section{Memberships}
\begin{tabular}{ll}
2011--Present & IEEE Member \\
2011--Present & IEEE Power and Energy Society Member \\
2010--Present & UK Magnetics Society Member\\
2005--Present & TMMOB Electrical Engineers Society Member\\
\end{tabular}

\section{Academic Work}
\begin{tabular}{ll}
Reviews for & IEEE Transactions of Industrial Electronics \\
& IEEE Transactions of Industrial Informatics \\
& IET Renewable Power Generation Journal \\
& IEEE International Conference on Electrical Machines and Drives \\
& IET Power Electronics, Machines and Drives Conference \\
& IEEE International Conference on Electrical Machines\\
\end{tabular}


\section{Software}
\begin{itemize}
\item Programming: Python, R, Matlab, C++.
\item Finite Element Analysis: Opera, FEMM, ANSYS, Gmsh, GetDP.
\item Computer Aided Design: SolidWorks, AutoCad.
\item Computational Fluid Dynamics: OpenFOAM.
\end{itemize}

\section{Activities \& Hobbies}
\begin{tabular}{ll}
Mountaineering & Active Member of METU Mountaineering Club\\
Sailing & Dinghy and Yacht Sailing, Certificate of Competence for Operators of Pleasure Craft. \\
Blogging & asuyatuyolar.org\\
\end{tabular}

\section{Languages}
\begin{tabular}{lll} 
English (Fluent), & Spanish (Beginner), & Chinese (Beginner)
\end{tabular}


\begin{publications}

I have published 6 journal papers, 13 referred conference papers (2 of which won best paper/poster award), have filed one patent and contributed to two books published by Woodhead Pub. and the IET. My papers have been cited 65 times so far, one paper, which has been included in RAE for UoE, has been cited 20 times.

Link: \url{http://ozan.keysan.me/papers}

%\printbib{posters}
%\printbib{books}
\printbib{papers}
\printbib{chapters}
\printbib{patent}
\printbib{conferences}
\printbib{conference-abstract}
\printbib{national}
%\printbib{reports}
\printbib{thesis}
%\printbib{unpublished}
%\printbib{techreports}
%\printbib{bookreviews}
%\printbib{editorials}
\end{publications}

%Each \printbib command will add a section with all the publications in that cateegory,
%listed in chronological order and sorted by name within each year.
%The total number of publications listed inside the publications environment is 
%calculated and the page numbers for the publications sections are stored. 
%So I have the following line in the Research section of my CV:

%I have authored \ref{sumpapers} papers, chapters or books on statistical topics. A list of these appears on pages \pageref{papersstart}--\pageref{papersend}.

\end{document}
