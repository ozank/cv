\documentclass[a4paper,12pt]{article}
%\usepackage[backend=biber]{biblatex}
\usepackage{cv_ozi_tr}


\usepackage{libertine}
%\usepackage{librebaskerville}

%\usepackage[sc]{mathpazo}
%\linespread{1.05}  

\usepackage[T1]{fontenc}

\usepackage{longtable}
%\geometry{top=18mm, inner=20mm, outer=20mm, bottom=00mm}
%\geometry{top=18mm, left=12mm, right=15mm, bottom=00mm}
%\geometry{text={17.6cm,26.3cm},centering}
%\titlespacing{\section}{0pt}{1.20ex}{0.5ex}


\name{Ozan Keysan}

\info{Address: & Orta Doğu Teknik Üniversitesi\\
	& Elektrik-Elektronik Müh., 06800, Ankara\\
      Phone: & +90 312 210 23 19\\
      Email: & keysan@metu.edu.tr\\
      Web: & http://.keysan.me} %\\
%      Marital status: & single}
%      WWW: & ToBeIncluded.com} 
 
\clearpage

\bibliography{ozan_publications}
\makebibcategory{national}{Ulusal Yayınlar}
\makebibcategory{patent}{Patent}
\makebibcategory{thesis}{Tez}
\makebibcategory{reports}{Teknik Raporlar}
 
%\addtocategory{books}{MWH3,ITSM91,ITSM94,expsmooth08}
\addtocategory{papers}{Keysan2009,Hodgins2011a,Keysan2011b,Keysan2012a,Keysan2012e,Keysan2012d,Keysan2014e}
%\addtocategory{papers}{HDR96,HBG96,HF96,GHH97,HW97,LFSH97,GH98}

%EWEC2015 ekle
\addtocategory{conferences}{Hodgins2009,Hodgins2010,Keysan2010c,Keysan2010b,Keysan2010a,Keysan2010,Keysan2011e,Keysan2011c,Keysan2011d,Keysan2012,Echenique2013, Keysan2013a,Keysan2013b,Keysan2014,Keysan2014a}
\addtocategory{chapters}{Keysan2013,Keysan2014f}
\addtocategory{patent}{Keysan2011g}
\addtocategory{national}{Ertan2009a,Keysan2012c,Ertan2012}
\addtocategory{thesis}{Keysan2008a,Keysan2014c}
\addtocategory{reports}{Keysan2012g,Keysan2012f,Keysan2010e,Keysan2009h,Keysan2009g,Keysan2009f,Keysan2009e}
\addtocategory{unpublished}{Mueller2013,Lara2013}


%books 	Books
%papers 	Ref­er­eed research papers
%chap­ters 	Book chap­ters
%con­fer­ences 	Papers in con­fer­ence proceedings
%techre­ports 	Unpub­lished work­ing papers
%bookre­views 	Book reviews
%edi­to­ri­als 	Edi­to­ri­als
%phd 	PhD the­sis
%sub­pa­pers 	Sub­mit­ted papers
%cur­pa­pers 	Cur­rent projects

%It is easy to add your own categories and titles if these are not suitable. 
%For example, if you want to include posters in your CV, put the following in the preamble:

%\makebibcategory{posters}{Conference posters}
%\makebibcategory{posters}{Publications}

%\addtocategory{posters}{Portugal2012,McDonald2008f}

%After the preamble, my CV looks like this: 
\begin{document}
\maketitle

\section{İş Tecrübesi}

\begin{tabular}{lp{3.6cm}l}
2014-- & Yardımcı Doçent & ODTÜ Elektrik-Elektronik Mühendisliği \\
2011--2014 & Araştırma Görevlisi & Enerji Sistemleri Enstitüsü, Edinburgh Üniversitesi \\ 
2010--2012 & Tasarım Danışmanı & NGenTec, Edinburgh\\
2009--2010 & Araştırma Görevlisi & Enerji Sistemleri Enstitüsü, Edinburgh Üniversitesi \\ 
2005--2009 & Araştırma Görevlisi & ODTÜ Elektrik-Elektronik Mühendisliği \\
\end{tabular}
 
\section{Eğitim}
\begin{tabular}{llp{12cm}}
2013 & Doktora & Edinburgh  Üniversitesi, Enerji Sistemleri Enstitüsü, İngiltere\\  
& & \textit{Tez Başlığı: Büyük Rüzgar Türbinleri için Süperiletken Jenerator Tasarımı} \\ 
2008 & Yüksek Lisans & ODTÜ Elektrik-Elektronik Mühendisliği\\
& & \textit{Tez Başlığı: Asenkron Motorlar için Harici Bobin ile Modifiyesiz Hız Ve Pozisyon Algılayıcısı}\\
2005 & Lisans & ODTÜ Elektrik-Elektronik Mühendisliği \\

\end{tabular}


% \section{Research Interests}

% My main area of interest is in the design and development of electrical machines. In particular, I am working on novel machine topologies such as superconducting machines and permanent-magnet machines. I also worked on the structural and thermal analysis of electrical machines and implementation of linear machines to wave energy converters.

\section{Araştırma Projeleri}
\begin{longtable}{lp{3cm}p{12cm}}
2011-- & Avrupa Birliği 7. Çerçeve Programı & Marina Platform Projesi'nde halen tam zamanlı araştırmacı. Projenin amacı dalga ve rüzgar enerjisini birleştiren yüzer platformlar tasarlamaktır.\\
2013 & Birleşik Arap Emirlikleri Üniversitesi & 5 kW'lık sabit mıknatıslı doğrusal jeneratör tasarımı.\\
2013 & General Electric & Doktora tezim sırasında geliştirdiğim süperiletken jeneratör tasarımının laboratuar testleri. Süperiletken bobin ve test altyapısı General Electric tarafından sağlanmıştır.\\
2011 & NGenTec & 5 MW, 300 rpm sabit mıknatıslı jeneratör tasarım danışmanlığı ve sonlu element analizi.\\
2010 & NGenTec & 1 MW, 12 rpm sabit mıknatıslı jeneratör tasarımı. Jeneratör başarıyla üretilmiş ve test edilmiştir.\\
2010 & SMART R\&D Grant & 25 kW eksenel sabit mıknatıslı jeneratörün tasarım ve üretimi.\\
2010 & Hayward Tyler & Su altında çalışacak bir sabit mıknatıs motor tasarımı. Motorun termal performansı, aşınma mekanizmaları ve sürtünme kayıpları incelenmiştir.\\
2009--2010 & NPower Project & Dalga ve Gel-git enerji sistemlerinde kullanılmak üzere sabit mıknatıslı jeneratörlerin fizibilite çalışması. Dört firma ile (Aquamarine, AWS Ocean Power, Marine Current Turbines, Scotrenewables) ortaklaşa çalışılmıştır. Edinburgh Üniversitesi tarafından lisanslanan bir analitik tasarım ve optimizasyon programı geliştirilmiştir.\\
2005--2008 & ODTÜ & Motor dışına taşan manyetik akıyı kullanarak rotor hız ve pozisyonunun tespiti. Yöntem ile ilgili uluslararası patent alınmıştır.\\
\end{longtable}

% \section{Ders Tecrübesi}

% \subsection{School of Engineering, University of Edinburgh}

% Laboratory Supervision:
% \begin{itemize}
% \item Power Engineering Lab (2nd year): This course introduces students to the techniques and equipment used in the generation, transmission, distribution and utilisation of electrical power.
% \item Power Generation Lab (3rd year): The lab aims to give the students experience in working with rotating machines and power electronic equipment and synchronization to the grid. 

% \end{itemize}

% Courses:
% \begin{itemize}
% \item The Industrial Doctoral Centre for Offshore Renewable Energy (four-year EngD programme), "Introduction to Superconductivity and Superconducting Generators".
% \end{itemize}


% M.Sc. Supervision:
% \begin{itemize}
% \item Mario Recio Lara, "Development of a mobile phone application to detect speed and faults of electrical machines", 2013--present.
% \item Marzia Akbari, "Comparison and control of power take-off systems for combined wind/wave energy platforms", 2013--present.
% \end{itemize}



% \subsection{Middle East Technical University}
% Laboratory Supervision
% \begin{itemize}
% \item Electromechanical Energy Conversion I-II (EE361, EE362, undergraduate 3rd year).
% \item Static Power Conversion (EE463, undergraduate 4th year).
% \item Laboratory coordinator (4 semesters). 
% \end{itemize}


\section{Burs \& Ödüller}
\begin{tabular}{lp{16cm}}
%2013 & Staff Scholarship, University of Edinburgh. \\
2012 & Genç Araştırmacı Desteği, International Conference on Superconductivity and Magnetism, ICSM.\\ 
2011 & Genç Araştırmacı Ödülü ve Seyahat Bursu, European Conference on Applied Superconductivity, EUCAS. \\ 
2011 & IEEE Ödülü ve Seyahat Bursu, IEEE Power Electronics Society.\\ 
2011 & En İyi Poster Ödülü, IEEE International Electric Machines and Drives Conference, IEMDC.\\
2010 & En İyi Makale Ödülü, IEEE International Conference on Electrical Machines, ICEM. \\
2012-2013 & Doktora Bursu, Edinburgh Üniversitesi. \\
2010-2011 & Yenilenebilir Enerji Bursu, Hopewell Holdings, Hong Kong. \\
2005--2007 & Yüksek Lisans Bursu, TÜBİTAK. \\
2005 & ALES Türkiye ikincilik derecesi.\\
2005 & ODTÜ Yeni İşler Yeni Fikirler Finalisti.\\
2001--2005 & ODTÜ Yüksek Şeref listesi(3 dönem), Şeref Listesi (3 dönem). \\
\end{tabular}


% \section{Kurs \& Seminerler}
% \begin{itemize}
% \item Future Reliable Renewable Energy Conversion Systems, 4th Flagship Seminar, Chongqing, China, 2012.
% %\item Supervising Postgraduate Research, Iain Davidson, 2013. 
% \item Large-Scale Parallel Computing, University of Edinburgh, 2013.
% \item Superconducting Machines, UK Magnetics Society, University of Oxford, 2012.
% \item Thermal and Mechanical Aspects of High Performance Electrical Machines, UK Magnetics Society, 2011.
% \item LaTeX for scientific publications, Skills Development Edinburgh, 2010.
% \item Opera 2D/3D FEA Analysis of Electrical Machines, Cobham Ltd. Vector Fields, 2009.
% \item Marine and Tidal Energy Workshop, Industrial Problems in Marine Energy Workshop, 2009.
% \item Project Management for Researchers, Fistral Training \& Consultancy Ltd., 2009.
% \item Project Presentation on International Venture Capital Forum, Athens, 2006.
% \item Project Management and Entrepreneurship Course, 30 days, METU Technopolis, 2005.
% \end{itemize}


\section{Üyelikler}
\begin{tabular}{ll}
2011-- & IEEE\\
2011-- & IEEE Güç ve Enerji Topluluğu\\
2010-- & UK Manyetizma Topluluğu\\
2005-- & TMMOB Elektrik Mühendisleri Odası\\
\end{tabular}

\section{Akademik Çalışmalar}
\begin{tabular}{ll}
Hakemlik & IEEE Transactions of Industrial Electronics \\
& IEEE Transactions of Industrial Informatics \\
& IET Renewable Power Generation Journal \\
& IEEE International Conference on Electrical Machines and Drives \\
& IET Power Electronics, Machines and Drives Conference \\
& IEEE International Conference on Electrical Machines\\
\end{tabular}


\section{Yazılım}
\begin{itemize}
\item Programlama: Python, R, Matlab, C++.
\item Sonlu Eleman Analizi: Opera, FEMM, ANSYS, Gmsh, GetDP.
\item CAD: SolidWorks, AutoCad.
%\item Computational Fluid Dynamics (CFD), OpenFOAM.
\end{itemize}

% \section{Activities \& Hobbies}
% \begin{tabular}{ll}
% Mountaineering & Active Member of METU Mountaineering Club\\
% Sailing & Dinghy and Yacht Sailing, Certificate of Competence For Operators of Pleasure Craft. \\
% Blogging & asuyatuyolar.org\\
% \end{tabular}

\section{Yabancı Dil}
\begin{tabular}{lll} 
İngilizce (Akıcı), & İspanyolca (Orta düzey), & Çince(Mandarin) (Başlangıç)
\end{tabular}

\clearpage

\begin{publications}
%\printbib{posters}
%\printbib{books}
\printbib{papers}
\printbib{chapters}
\printbib{patent}
\printbib{conferences}
\printbib{national}
\printbib{reports}
\printbib{thesis}
\printbib{unpublished}
%\printbib{techreports}
%\printbib{bookreviews}
%\printbib{editorials}
\end{publications}

%Each \printbib command will add a section with all the publications in that cateegory,
%listed in chronological order and sorted by name within each year.
%The total number of publications listed inside the publications environment is 
%calculated and the page numbers for the publications sections are stored. 
%So I have the following line in the Research section of my CV:

%I have authored \ref{sumpapers} papers, chapters or books on statistical topics. A list of these appears on pages \pageref{papersstart}--\pageref{papersend}.

\end{document}
